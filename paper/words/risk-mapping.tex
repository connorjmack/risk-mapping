%% RAI (Rockfall Activity Index) Literature
%% Bibliography file for PC-RAI project
%% Compatible with Zotero import
%% Last updated: January 2025

%% =============================================================================
%% CORE RAI METHODOLOGY PAPERS
%% =============================================================================

@article{dunham2017rockfall,
  title = {Rockfall {{Activity Index}} ({{RAI}}): {{A}} Lidar-Derived, Morphology-Based Method for Hazard Assessment},
  author = {Dunham, Lisa and Wartman, Joseph and Olsen, Michael J. and O'Banion, Matthew and Cunningham, Keith},
  year = {2017},
  journal = {Engineering Geology},
  volume = {221},
  pages = {184--192},
  doi = {10.1016/j.enggeo.2017.03.009},
  abstract = {In this paper, we introduce the Rockfall Activity Index (RAI), a point cloud-derived, high-resolution, morphology-based approach for assessing rockfall hazards. With the RAI methodology, rockfall hazards are evaluated in a two-step procedure. First, morphological indices (local slope and roughness) are used to classify mass wasting processes acting on a rock-slope. These classifications are then used with estimated instability rates to map rockfall activity across an entire slope face. The rockfall hazard is quantified as the estimated annual kinetic energy produced by rockfall along 1-m length segments of a slope face.},
  keywords = {RAI, rockfall, lidar, hazard assessment, point cloud, morphology}
}

@article{markus2023lidar,
  title = {Lidar-{{Derived Rockfall Inventory}}---{{An Analysis}} of the {{Geomorphic Evolution}} of {{Rock Slopes}} and {{Modifying}} the {{Rockfall Activity Index}} ({{RAI}})},
  author = {Markus, Shane J. and Wartman, Joseph and Olsen, Michael and Darrow, Margaret M.},
  year = {2023},
  journal = {Remote Sensing},
  volume = {15},
  number = {17},
  pages = {4223},
  doi = {10.3390/rs15174223},
  abstract = {Rockfall presents a significant risk to the safety and economy of communities and infrastructure in mountainous regions. The recently-developed Rockfall Activity Index (RAI) utilizes high-resolution terrestrial lidar-derived digital elevation models (DEMs) of rock slopes to categorize a slope face into seven distinct morphological units, or "RAI classes". This paper focuses on a comprehensive study conducted at four sites in Alaska, USA, where a robust lidar-based five-year inventory of 4381 rockfall events was analyzed.},
  keywords = {RAI, rockfall inventory, lidar, change detection, Alaska, magnitude-frequency}
}

@mastersthesis{dunham2015rockslope,
  title = {Rock-Slope {{Activity Index}} ({{RAI}}): {{A}} Lidar-Derived Process-Based Rock-Slope Assessment System},
  author = {Dunham, Lisa Ann},
  year = {2015},
  school = {University of Washington},
  address = {Seattle, WA},
  type = {Master's {{Thesis}}},
  abstract = {Inventory of unstable highway slopes is an immense challenge for Departments of Transportation (DOTs) due to the geographic dispersion of problematic slopes as well as the variable nature and speed of erosional processes. Due to advancements in lidar technology, acquisition of high resolution spatial data to map and monitor these slopes is becoming simpler, less expensive, and more widely available.}
}

%% =============================================================================
%% RAI APPLICATIONS AND EXTENSIONS
%% =============================================================================

@article{olsen2023forecasting,
  title = {Forecasting Post-Earthquake Rockfall Activity},
  author = {Olsen, Michael J. and Massey, Chris and Leshchinsky, Ben and Wartman, Joseph and Senogles, Andrew},
  year = {2023},
  journal = {Journal of Applied Geodesy},
  volume = {17},
  number = {2},
  pages = {171--179},
  doi = {10.1515/jag-2022-0045},
  abstract = {This paper interrogates a rich database of repeat terrestrial lidar scans collected during the Canterbury New Zealand Earthquake Sequence to document geomorphic processes as well as quantify rockfall activity rates through time. Changes in the activity rate (spatial distribution) and failure depths (size) were observed based on the Rockfall Activity Index (RAI) morphological classification.},
  keywords = {earthquakes, erosion, lidar, monitoring, rockslope, RAI, New Zealand}
}

@techreport{olsen2020predicting,
  title = {Predicting {{Seismically Induced Rockfall Hazard}} for {{Targeted Site Mitigation}}},
  author = {Olsen, Michael J. and Massey, Chris and Leshchinsky, Ben A. and Senogles, Andrew and Wartman, Joseph},
  year = {2020},
  institution = {Oregon Department of Transportation Research},
  address = {Salem, OR},
  number = {SPR-809},
  abstract = {This research develops a method to predict seismic rockfall areas by integrating lidar-based analysis with the Rockfall Activity Index (RAI) methodology. The project produced RAMBO (Rockfall Activity Morphological Bigdata Optimizer) software and analyzed five high-risk ODOT highway sections.},
  keywords = {ODOT, seismic, rockfall, RAI, RAMBO, Oregon}
}

@techreport{holtan2023schmidt,
  title = {Using the {{Schmidt Hammer}} to {{Improve}} the {{Rockfall Activity Index}}},
  author = {Holtan, Kat and Olsen, Michael J. and Wartman, Joseph and Leshchinsky, Ben A.},
  year = {2023},
  institution = {Pacific Northwest Transportation Consortium (PacTrans)},
  url = {https://rosap.ntl.bts.gov/view/dot/73103},
  abstract = {This research correlates RAI classes with rock strength measurements using Schmidt hammer testing to improve and refine the accuracy and interpretation of RAI analysis for wider adoption by transportation authorities.},
  keywords = {RAI, Schmidt hammer, rock strength, PacTrans}
}

@techreport{cunningham2021quantifying,
  title = {Quantifying the {{Impact}} of {{Rockfall}} on the {{Mobility}} of {{Critical Transportation Corridors}}},
  author = {Cunningham, Keith W. and Leshchinsky, Ben A. and Olsen, Michael J. and Wartman, Joseph and Shaefer, K.},
  year = {2021},
  institution = {Pacific Northwest Transportation Consortium (PacTrans)},
  abstract = {This research extends the Rockfall Activity Rate System (RoARS) model to evaluate co-seismic and post-seismic rockfall hazard at a regional scale in Alaska transportation corridors.},
  keywords = {RoARS, rockfall, Alaska, transportation, seismic}
}

@techreport{darrow2023longterm,
  title = {The {{Long-Term Effect}} of {{Earthquakes}}: {{Using Geospatial Solutions}} to {{Evaluate Heightened Rockfall Activity}} on {{Critical Lifelines}}},
  author = {Darrow, Margaret M. and Leshchinsky, Ben A. and Olsen, Michael J. and Wartman, Joseph},
  year = {2023},
  institution = {Pacific Northwest Transportation Consortium (PacTrans)},
  url = {https://rosap.ntl.bts.gov/view/dot/72718},
  abstract = {This research extends the Rockfall Activity Rate System (RoARS) model to evaluate co-seismic and post-seismic rockfall hazard at a regional scale in Alaska transportation corridors.},
  keywords = {RoARS, rockfall, Alaska, transportation, seismic, earthquakes}
}

%% =============================================================================
%% RELATED ROCKFALL SUSCEPTIBILITY METHODS
%% =============================================================================

@article{matasci2018assessing,
  title = {Assessing Rockfall Susceptibility in Steep and Overhanging Slopes Using Three-Dimensional Analysis of Failure Mechanisms},
  author = {Matasci, Battista and Stock, Greg M. and Jaboyedoff, Michel and Carrea, Dario and Collins, Brian D. and Gu{\'e}rin, Antoine and Matasci, Gregoire and Ravanel, Ludovic},
  year = {2018},
  journal = {Landslides},
  volume = {15},
  number = {5},
  pages = {859--878},
  doi = {10.1007/s10346-017-0911-y},
  abstract = {Rockfalls strongly influence the evolution of steep rocky landscapes and represent a significant hazard in mountainous areas. This paper presents methods for quantifying rockfall susceptibility at the cliff scale using three-dimensional point clouds to compute failure mechanisms for planar, wedge, and toppling failures on vertical and overhanging rock walls.},
  keywords = {rockfall susceptibility, 3D point cloud, failure mechanisms, overhanging slopes, Yosemite}
}

@article{farmakis2021identification,
  title = {Identification of Potential Rockfall Sources Using {{UAV}}-Derived Point Cloud},
  author = {Farmakis, Ioannis and Marinos, Vassilis and Papathanassiou, George and Karantanellis, Efstratios},
  year = {2021},
  journal = {Bulletin of Engineering Geology and the Environment},
  volume = {80},
  number = {8},
  pages = {6085--6103},
  doi = {10.1007/s10064-021-02306-2},
  abstract = {This paper presents a novel methodology to use the extracted discontinuity set characteristics for a local scale rockfall susceptibility assessment, tailored for Uncrewed Aerial Vehicle (UAV) data acquisition. The method develops a new Rockfall Susceptibility Index combining discontinuity analysis with slope stability assessment.},
  keywords = {UAV, rockfall susceptibility, point cloud, discontinuity, RSI}
}

@article{gigli2014terrestrial,
  title = {Terrestrial Laser Scanner and Geomechanical Surveys for the Rapid Evaluation of Rock Fall Susceptibility Scenarios},
  author = {Gigli, Giovanni and Morelli, Stefano and Fornera, Silvia and Casagli, Nicola},
  year = {2014},
  journal = {Landslides},
  volume = {11},
  number = {1},
  pages = {1--14},
  doi = {10.1007/s10346-012-0374-0},
  keywords = {TLS, rockfall, susceptibility, geomechanics}
}

%% =============================================================================
%% LIDAR CHANGE DETECTION AND POINT CLOUD PROCESSING
%% =============================================================================

@article{olsen2015fill,
  title = {To {{Fill}} or {{Not}} to {{Fill}}: {{Sensitivity Analysis}} of the {{Influence}} of {{Resolution}} and {{Hole Filling}} on {{Point Cloud Surface Modeling}} and {{Individual Rockfall Event Detection}}},
  author = {Olsen, Michael J. and Wartman, Joseph and McAlister, M. and Mahmoudabadi, H. and O'Banion, Matthew S. and Dunham, Lisa and Cunningham, Keith},
  year = {2015},
  journal = {Remote Sensing},
  volume = {7},
  number = {9},
  pages = {12103--12134},
  doi = {10.3390/rs70912103},
  abstract = {This paper presents a sensitivity analysis of the influence of resolution and hole filling on point cloud surface modeling and individual rockfall event detection for lidar-based rock slope assessments.},
  keywords = {lidar, point cloud, hole filling, resolution, rockfall detection}
}

@article{lague2013accurate,
  title = {Accurate {{3D}} Comparison of Complex Topography with Terrestrial Laser Scanner: {{Application}} to the {{Rangitikei}} Canyon ({{N}}-{{Z}})},
  author = {Lague, Dimitri and Brodu, Nicolas and Leroux, J{\'e}r{\^o}me},
  year = {2013},
  journal = {ISPRS Journal of Photogrammetry and Remote Sensing},
  volume = {82},
  pages = {10--26},
  doi = {10.1016/j.isprsjprs.2013.04.009},
  abstract = {This paper presents M3C2, a method for accurate 3D comparison of point clouds that handles complex topography and different point densities.},
  keywords = {M3C2, point cloud comparison, change detection, TLS}
}

@article{brodu2012lidar,
  title = {{{3D}} Terrestrial Lidar Data Classification of Complex Natural Scenes Using a Multi-Scale Dimensionality Criterion: {{Applications}} in Geomorphology},
  author = {Brodu, Nicolas and Lague, Dimitri},
  year = {2012},
  journal = {ISPRS Journal of Photogrammetry and Remote Sensing},
  volume = {68},
  pages = {121--134},
  doi = {10.1016/j.isprsjprs.2012.01.006},
  abstract = {This paper presents CANUPO, a method for 3D point cloud classification using multi-scale dimensionality analysis.},
  keywords = {CANUPO, point cloud classification, multi-scale, geomorphology}
}

@article{rosser2005terrestrial,
  title = {Terrestrial Laser Scanning for Monitoring the Process of Hard Rock Coastal Cliff Erosion},
  author = {Rosser, N. J. and Petley, D. N. and Lim, M. and Dunning, S. A. and Allison, R. J.},
  year = {2005},
  journal = {Quarterly Journal of Engineering Geology and Hydrogeology},
  volume = {38},
  number = {4},
  pages = {363--375},
  doi = {10.1144/1470-9236/05-008},
  abstract = {This paper demonstrates the application of terrestrial laser scanning for monitoring hard rock coastal cliff erosion processes.},
  keywords = {TLS, coastal cliff, erosion, monitoring}
}

%% =============================================================================
%% DISCONTINUITY AND ROCK MASS CHARACTERIZATION
%% =============================================================================

@article{gigli2011semiautomatic,
  title = {Semi-Automatic Extraction of Rock Mass Structural Data from High Resolution {{LIDAR}} Point Clouds},
  author = {Gigli, Giovanni and Casagli, Nicola},
  year = {2011},
  journal = {International Journal of Rock Mechanics and Mining Sciences},
  volume = {48},
  number = {2},
  pages = {187--198},
  doi = {10.1016/j.ijrmms.2010.11.009},
  keywords = {lidar, rock mass, discontinuity extraction, semi-automatic}
}

@article{lato2012automated,
  title = {Automated Mapping of Rock Discontinuities in {{3D}} Lidar and Photogrammetry Models},
  author = {Lato, Matthew J. and V{\"o}ge, Malte},
  year = {2012},
  journal = {International Journal of Rock Mechanics and Mining Sciences},
  volume = {54},
  pages = {150--158},
  doi = {10.1016/j.ijrmms.2012.06.003},
  keywords = {discontinuity mapping, lidar, photogrammetry, automated}
}

@article{riquelme2014new,
  title = {A New Approach for Semi-Automatic Rock Mass Joints Recognition from {{3D}} Point Clouds},
  author = {Riquelme, Adri{\'a}n J. and Abell{\'a}n, Antonio and Tom{\'a}s, Roberto and Jaboyedoff, Michel},
  year = {2014},
  journal = {Computers \& Geosciences},
  volume = {68},
  pages = {38--52},
  doi = {10.1016/j.cageo.2014.03.014},
  abstract = {This paper presents Discontinuity Set Extractor (DSE), a semi-automatic approach for rock mass joint recognition from 3D point clouds.},
  keywords = {DSE, discontinuity, point cloud, rock mass, joints}
}

@inproceedings{dewez2016facets,
  title = {Facets: {{A CloudCompare}} Plugin to Extract Geological Planes from Unstructured {{3D}} Point Clouds},
  author = {Dewez, Thomas J. B. and Girardeau-Montaut, Daniel and Allanic, C{\'e}cile and Rohmer, J{\'e}r{\'e}my},
  year = {2016},
  booktitle = {International {{Archives}} of the {{Photogrammetry}}, {{Remote Sensing}} and {{Spatial Information Sciences}}},
  volume = {XLI-B5},
  pages = {799--804},
  doi = {10.5194/isprs-archives-XLI-B5-799-2016},
  keywords = {CloudCompare, Facets, geological planes, point cloud}
}

@article{tannant2022critical,
  title = {A Critical Review of Discontinuity Plane Extraction from {{3D}} Point Cloud Data of Rock Mass Surfaces},
  author = {Riquelme, Adri{\'a}n J. and Tannant, Dwayne D. and Tom{\'a}s, Roberto and Abell{\'a}n, Antonio},
  year = {2022},
  journal = {Computers \& Geosciences},
  volume = {169},
  pages = {105241},
  doi = {10.1016/j.cageo.2022.105241},
  abstract = {This paper reviews the capabilities, merits, and limitations of different segmentation methods for discontinuity plane surface extraction from point cloud data collected from rock faces.},
  keywords = {discontinuity extraction, point cloud, review, segmentation}
}

@article{chen2024impact,
  title = {Impact of Discontinuity Data Acquisition Methods on Rockfall Susceptibility Assessment Using High-Resolution {{3D}} Point Cloud},
  author = {Gigli, Giovanni and Casagli, Nicola and others},
  year = {2024},
  journal = {Engineering Geology},
  volume = {309},
  pages = {107571},
  doi = {10.1016/j.enggeo.2024.107571},
  abstract = {This research evaluates the influence of various discontinuity data acquisition methods on three-dimensional rockfall susceptibility assessments using high-resolution 3D point clouds.},
  keywords = {discontinuity, rockfall susceptibility, 3D point cloud, KHI}
}

%% =============================================================================
%% MAGNITUDE-FREQUENCY AND ROCKFALL STATISTICS
%% =============================================================================

@article{dussauge2003statistical,
  title = {Statistical Analysis of Rockfall Volume Distributions: {{Implications}} for Rockfall Dynamics},
  author = {Dussauge, C{\'e}cile and Grasso, Jean-Robert and Helmstetter, Agn{\`e}s},
  year = {2003},
  journal = {Journal of Geophysical Research: Solid Earth},
  volume = {108},
  number = {B6},
  doi = {10.1029/2001JB000650},
  abstract = {This paper presents statistical analysis of rockfall volume distributions showing power-law behavior.},
  keywords = {rockfall, magnitude-frequency, power law, statistics}
}

@article{williams2019importance,
  title = {The {{Importance}} of {{Monitoring Interval}} for {{Rockfall Magnitude}}-{{Frequency Estimation}}},
  author = {Williams, J. G. and Rosser, N. J. and Hardy, R. J. and Brain, M. J.},
  year = {2019},
  journal = {Journal of Geophysical Research: Earth Surface},
  volume = {124},
  number = {12},
  pages = {2841--2853},
  doi = {10.1029/2019JF005225},
  keywords = {rockfall, magnitude-frequency, monitoring interval}
}

@article{malamud2004landslide,
  title = {Landslide Inventories and Their Statistical Properties},
  author = {Malamud, Bruce D. and Turcotte, Donald L. and Guzzetti, Fausto and Reichenbach, Paola},
  year = {2004},
  journal = {Earth Surface Processes and Landforms},
  volume = {29},
  number = {6},
  pages = {687--711},
  doi = {10.1002/esp.1064},
  keywords = {landslide inventory, statistics, power law}
}

%% =============================================================================
%% GENERAL LIDAR AND REMOTE SENSING REVIEWS
%% =============================================================================

@article{jaboyedoff2012use,
  title = {Use of {{LIDAR}} in Landslide Investigations: {{A}} Review},
  author = {Jaboyedoff, Michel and Oppikofer, Thierry and Abell{\'a}n, Antonio and Derron, Marc-Henri and Loye, Alex and Metzger, Richard and Pedrazzini, Andrea},
  year = {2012},
  journal = {Natural Hazards},
  volume = {61},
  number = {1},
  pages = {5--28},
  doi = {10.1007/s11069-010-9634-2},
  abstract = {This paper reviews the use of LIDAR technology in landslide investigations including rockfall assessment.},
  keywords = {lidar, landslide, review, remote sensing}
}

@article{lato2012evaluating,
  title = {Evaluating Roadside Rockmasses for Rockfall Hazards Using {{LiDAR}} Data: {{Optimizing}} Data Collection and Processing Protocols},
  author = {Lato, Matthew J. and Diederichs, Mark S. and Hutchinson, D. Jean and Harrap, Rob},
  year = {2012},
  journal = {Natural Hazards},
  volume = {60},
  number = {3},
  pages = {831--864},
  doi = {10.1007/s11069-011-9872-y},
  keywords = {lidar, rockfall hazard, roadside, data collection}
}

@article{abellan2014terrestrial,
  title = {Terrestrial Laser Scanning of Rock Slope Instabilities},
  author = {Abell{\'a}n, Antonio and Oppikofer, Thierry and Jaboyedoff, Michel and Rosser, Nick J. and Lim, Michael and Lato, Matthew J.},
  year = {2014},
  journal = {Earth Surface Processes and Landforms},
  volume = {39},
  number = {1},
  pages = {80--97},
  doi = {10.1002/esp.3493},
  keywords = {TLS, rock slope, instability, review}
}

%% =============================================================================
%% COASTAL EROSION (RELATED APPLICATIONS)
%% =============================================================================

@article{olsen2011new,
  title = {New Automated Point-Cloud Alignment for Ground-Based Light Detection and Ranging Data of Long Coastal Sections},
  author = {Olsen, Michael J. and Johnstone, E. and Kuester, F. and Driscoll, N. and Ashford, S. A.},
  year = {2011},
  journal = {Journal of Surveying Engineering},
  volume = {137},
  number = {1},
  pages = {14--25},
  doi = {10.1061/(ASCE)SU.1943-5428.0000030},
  keywords = {lidar, coastal, point cloud alignment, ground-based}
}

@article{olsen2013hinged,
  title = {Hinged, {{Pseudo}}-{{Grid Triangulation Method}} for {{Long}}, {{Near}}-{{Linear Cliff Analyses}}},
  author = {Olsen, Michael J. and Kuester, F. and Johnstone, E.},
  year = {2013},
  journal = {Journal of Surveying Engineering},
  volume = {139},
  number = {3},
  pages = {105--109},
  doi = {10.1061/(ASCE)SU.1943-5428.0000103},
  abstract = {This paper presents a hinged, pseudo-grid triangulation method for processing long, near-linear cliff sections from lidar data.},
  keywords = {cliff analysis, triangulation, lidar, coastal}
}

%% =============================================================================
%% SOFTWARE AND TOOLS
%% =============================================================================

@software{cloudcompare2023,
  title = {{{CloudCompare}}},
  author = {{CloudCompare Development Team}},
  year = {2023},
  version = {2.13},
  url = {https://www.cloudcompare.org/},
  note = {Open source 3D point cloud processing software}
}

@software{maptek2017isite,
  title = {{{I}}-{{Site Studio}}},
  author = {{Maptek}},
  year = {2017},
  version = {7.0},
  address = {Adelaide, Australia},
  note = {Commercial point cloud processing software}
}

%% =============================================================================
%% DEEP LEARNING AND MACHINE LEARNING APPROACHES
%% =============================================================================

@article{weidner2019classification,
  title = {Classification Methods for Point Clouds in Rock Slope Monitoring: {{A}} Novel Machine Learning Approach and Comparative Analysis},
  author = {Weidner, Luke and Walton, Gabriel and Kromer, Ryan},
  year = {2019},
  journal = {Engineering Geology},
  volume = {263},
  pages = {105326},
  doi = {10.1016/j.enggeo.2019.105326},
  keywords = {machine learning, point cloud classification, rock slope monitoring}
}

@article{chen2025deep,
  title = {A Robust Deep Learning Approach for Rock Discontinuity Identification from Large Scale {{3D}} Point Clouds},
  author = {Chen, X. and others},
  year = {2025},
  journal = {Scientific Reports},
  volume = {15},
  pages = {31137},
  doi = {10.1038/s41598-025-31137-4},
  abstract = {This study proposes RL-JointNet, an end-to-end deep learning approach for robust discontinuity segmentation from rock mass point clouds.},
  keywords = {deep learning, discontinuity, point cloud, segmentation}
}

%% =============================================================================
%% SEISMIC AND EARTHQUAKE-RELATED ROCKFALL
%% =============================================================================

@inproceedings{massey2015performance,
  title = {Performance of Rock Slopes during the 2010/11 {{Canterbury}} Earthquakes ({{New Zealand}})},
  author = {Massey, Chris I. and Richards, L. and Della-Pasqua, F. N. and McSaveney, M. J. and Holden, C. and Kaiser, A. E.},
  year = {2015},
  booktitle = {Proceedings of the 13th {{International Congress}} of {{Rock Mechanics}}: {{ISRM Congress}} 2015},
  publisher = {Canadian Institute of Mining, Metallurgy and Petroleum},
  address = {Montreal, Canada},
  note = {Paper No. 902},
  keywords = {rockfall, earthquake, Canterbury, New Zealand, rock slope}
}

@article{grant2017impact,
  title = {The Impact of Rockfalls on Dwellings during the 2011 {{Christchurch}}, {{New Zealand}} Earthquakes},
  author = {Grant, A. and Wartman, Joseph and Massey, Chris I. and Olsen, Michael J. and O'Banion, Matthew S. and Motley, M.},
  year = {2017},
  journal = {Landslides},
  doi = {10.1007/s10346-016-0758-7},
  keywords = {rockfall, earthquake, Christchurch, dwellings, impact}
}

%% =============================================================================
%% DATA REPOSITORIES
%% =============================================================================

@misc{markus2023designsafe,
  title = {Lidar-{{Derived Rockfall Inventory Dataset}} (2012-2017)},
  author = {Markus, Shane J. and Wartman, Joseph and Olsen, Michael and Darrow, Margaret M.},
  year = {2023},
  publisher = {DesignSafe-CI},
  url = {https://www.designsafe-ci.org/},
  note = {Openly accessible rockfall inventory database for four Alaska study sites}
}
